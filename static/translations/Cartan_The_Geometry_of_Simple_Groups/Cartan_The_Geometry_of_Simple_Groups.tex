%% LyX 2.3.6 created this file.  For more info, see http://www.lyx.org/.
%% Do not edit unless you really know what you are doing.
\documentclass[english]{article}
\usepackage[T1]{fontenc}
\usepackage[latin9]{inputenc}

\makeatletter
%%%%%%%%%%%%%%%%%%%%%%%%%%%%%% User specified LaTeX commands.
\usepackage{amsmath}

\makeatother

\usepackage{babel}
\begin{document}
\title{The Geometry of Simple Groups}
\author{�lie Cartan}
\maketitle

\subsection*{Introduction}

In a recent paper\footnote{{\'E}. Cartan, \emph{La G{\'e}om{\'e}trie des groupes de transformations}.
(J. Math. pures et appl., t. 6, 1927, pp.1-119).} developing and supplementing an earlier article published in collaboration
with J.A. Shouten\footnote{{\'E}. Cartan and J.A. Schouten, On the Geometry of the group-manifold
of simple and semi-simple groups. (Proc. Akad. Amsterdam, t. 29, 1926,
pp. 803-815).}, I studied the spaces with affine connection, without curvature or
torsion-free, representing continuous transformation groups. This
study applied to the most general groups, and it was local. In the
case of simple groups, the representative torsion-free spaces are
Riemannian, either complex or real (with $ds$ definite or indefinite).
Complex spaces represent simple groups with complex parameters; real
spaces represent simple groups with real parameters, unitary (if $ds$
is definite), or non-unitary (if $ds$ is indefinite). The last two
cases are distinguished from each other by the property that the space
is either closed or open.\\
\\
Riemann spaces representative of simple real unitary groups fall into
a more general and very important category of Riemann spaces, characterized
by the property that their Riemannian curvature is preserved by parallel
transport; their study is reduced to the study of those which I have
called \emph{irreducible} and which all relate to simple groups\footnote{The determination of all spaces is made in a paper, the first part
of which has just appeared. (Bull. Soc. Math. de France, t. 54, 1924,
pp. 214-264). See also {\'E}. Cartan,\emph{ Sur les espaces de Riemann
dans lesquels le transport par parall{\'e}lisme conserve la courbure}.
(Rend. Acc. Lincei, (6), t. $3^1$, 1926, pp. 544-547).}. In each of these irreducible spaces, the Riemannian curvature has
the same sign everywhere; in the same class there are both positive
curvature spaces and negative curvature spaces. Given a simple structure,
the representative spaces of the corresponding unitary real groups
have positive curvature; negative curvature spaces of the same class
are not representative of any group, but their group of displacements
is isomorphic to the group of the given structure with complex parameters.
If this is the simple structure with three parameters, the two spaces
are the three-dimensional spaces with constant curvature, positive
or negative; the first represents the group of rotations of ordinary
space; the group of displacements of the second is isomorphic to the
\emph{complex} homographic group of one variable.\\
\\
The detailed study of the most general irreducible Riemann spaces
is of great interest, both from the point of view of group theory
and from the geometric point of view. It will be the subject of a
later brief. In the present memoir I am only concerned with the two
particular classes mentioned above (spaces representative of simple
real unitary groups and their correspondents with negative curvature).
The study made here is no longer local; it relates rather to the properties
of space coming under the \emph{Analysis Situs}, to the distribution
of geodesics, to the complete determination of their mixed groups
of isometry, of their different Klein forms, etc. The questions which
arise are, moreover, of a very different nature depending on whether
the space has positive curvature or negative curvature.\\
\\
The first introductory chapter is devoted to the topology of simple
real unitary groups; it has its point of departure in the research
of Weyl relating to the theory of semi-simple groups\footnote{H. Weyl, \emph{Theorie der Darstellung kontinuierlicher halb-einfacher
Gruppen durch lineare Transformationen}. (Math. Zeitschr., t. 23,
1925, pp. 271-309; t. 24, 1925, pp. 328-395).}; the question is taken up in full; Weyl's results are complete and
the questions which arise are resolved to the end using one of my
recent papers\footnote{{\'E}. Cartan, \emph{Les tenseurs irr{\'e}ductibles et les groupes
lin{\'e}aires simples et semi-simples}. (Bull. Sc. Math., $2^{\text{nd}}$
series, t. 49, 1925, pp. 130-152).}\\
\\
Part II is devoted to the spaces of real unitary groups; the distribution
of geodesics is studied quite completely for the simply connected
forms of these spaces; it reveals, associated with each point in space,
the existence of a certain number of \emph{antipodal varieties} (which
can be reduced to points) which are in a way \emph{necked varieties}
for the closed geodesics resulting from the given point. There are
as many as units as the rank of the group.\\
\\
Part III is devoted to spaces with negative curvature whose group
of displacements has a simple complex structure. They are all just
related. We can thanks to them (and this is what will happen in all
the other irreducible spaces with negative curvature) to solve important
problems relating to their group of displacements. I will only point
out the following result: simple complex groups have, from the point
of view of Analysis Situs, the same properties as the corresponding
unitary real groups, and they always admit a simply connected linear
representative. This theorem itself results from a remarkable mode
of generation of finite transformations of the complex gorup: to take
just one example, each complex rotation of ordinary space is decomposable
in one way and only one into one. real and a rotation of a purely
imaginary angle around a real axis. Finally, the spaces in question
also have their importance from a purely geometric point of view,
but this importance will be manifested above all for the more general
irreducible spaces with negative curvature.\\
\\
I will assume that the fundamental principles of the theory of simple
groups are known\footnote{In this regard, the reader may refer to my Thesis (Paris, Nony, 1894)
or to the previously cited paper of Weyl; reading the paper cited
in footnote (${}^1$), will not be useless either.}.

\section*{Chapter I.}

\section*{The Topology of Simple Unitary Groups}

\subsubsection*{I. The Fundamental Polyhedron of the Adjoint Group}

\textbf{1.} We know\footnote{{\'E}. Cartan, \emph{Les tenseurs irr{\'e}ductibles et les groupes
lin{\'e}aires simples et semi-simples}. (Bull. Sc. Math., $2^{\text{nd}}$
series, t. 49, 1925, pp. 135).} that to each type of simple group of order $r$ belongs a real unitary
form, with $r$ real parameters, characterized by the property that
the sum of the squares of the characteristic roots of an arbitrary
infinitesimal transformation is a negative definite quadratic form
$-\phi(e)$. The unitary groups are, for the four major classes of
simple groups, respectively isomorphic:\\
\\
$A)$ to the unimodular linear group of a positive definite Hermitian
form \begin{equation*}
	x_1\bar x_1+x_2\bar x_2+\ldots+x_{l+1}\bar x_{l+1};
\end{equation*}$B)$ $D)$ to the linear group of a positive definite quadratic form
in $n=2l+1$ (type B) or $n=2l$ (type D) variables;\\
\\
$C)$ to the linear group leaving invariant a definite positive Hermitian
form; \begin{equation*}
	x_1\bar x_1+x_2\bar x_2+\ldots+x_{2l}\bar x_{2l};
\end{equation*}and an outer quadratic form \begin{equation*}
	[x_1 x_2]+[x_3 x_4]+\ldots+[x_{2l-1}x_{2l}].
\end{equation*} In the preceding notation, $l$ denotes the \emph{rank} of the group,
which will be discussed below.\\
\\
\textbf{2.} Let $G$ be a simple unitary group, $\Gamma$ its adjoint
group. $\Gamma$ is a linear group with $r$ real variables \begin{equation*}
	e_1,e_2,\ldots,e_r,
\end{equation*} which leaves the positive definite quadratic form $\phi(e)$ invariant.
Each transformation can be represented by a matrix $T$ of order $r$,
determinant equal to 1.\\
\\
Any matrix $T$ can, in an infinite number of ways, be generated by
an infinitesimal transformation $Y$ of $\Gamma$. Among the characteristic
roots of $Y$, $l$ are zero, the other $r-l$ are two-by-two equal
and opposite; they are linear combinations with integer coefficients
determined by $l$ of them (called \emph{fundamental}). All these
roots are purely imaginary: we will designate them, with Weyl, by
the notation \begin{equation*}
	2\pi i\phi_\alpha;
\end{equation*}the quantities $\phi_\alpha$ are the \emph{angular parameters} of
the transformation $Y$.\\
\\
The matrix $T$ generated by $Y$ admits $l$ characteristic roots
equal to 1, the others are the quantities $e^{2\pi i \phi_\alpha}$.\\
\\
\textbf{3.} If an infinitesimal transformation $Y$ is general, that
is to say does not admit more than $l$ characteristic zero roots,
there exist $l-1$ other infinitesimal transformations independent
of each other and independent of $Y$, which enjoy the property of
commuting among themselves and commuting with $Y$. We thus obtain
an Abelian subgroup $\gamma$ of order $l$. If \begin{equation*}
	e_1Y_1+e_2Y_2+\ldots+e_lY_l
\end{equation*}is the most general infinitesimal transformation of $\gamma$, the
angular parameters $\phi_\alpha$ are linear forms in $e_1,e_2,\ldots,e_l$
of which $l$ are linearly independent.\\
\\
If we now start from a singular infinitesimal transformation $Y$,
that is to say admitting more than $l$ characteristic zero roots,
there exists more than $l$ independent transformations exchangeable
with $Y$. We can ask ourselves if, among these transformations, there
exists at least one which is not singular. This is what we will demonstrate.\\
\\
Let $Y_1$ be a particular transformation commuting with $Y$; let
$Y_2$ be a particular transformations commuting with $Y$ and $Y_1$
(and linearly independent of $Y$ and $Y_1$), and so on. Suppose
that we can thus find $\lambda$ independent transformations \begin{equation*}
	Y,Y_1,\ldots,Y_{\lambda-1},
\end{equation*}commuting among themselves and such that no other transformation of
the group commutes at the same time with each of them. The characteristic
$\omega_\alpha$ roots of the transformation \begin{equation*}
	eY+e_1Y_1+\ldots+e_{\lambda-1}Y_{\lambda-1}
\end{equation*}are \emph{linear}\footnote{This is because the transformations commute with each other.}
forms in $e,e_1,\ldots,e_{\lambda-1}$. Among these linear forms,
there are at least $\lambda$ independent ones; otherwise there would
exist a non-zero transformation $\sum e_iY_i$ having all its characteristic
roots zero; this is impossible, since the sum of the squares of the
characteristic roots of an arbitrary transformation of the group is
a \emph{definite} form. Of the characteristic roots of an arbitrary
transformation a minimum of $\lambda$ are independant, this proves,
by the very definition of the rank, that $\lambda$ is at most equal
to $l$. But on the other hand $\lambda$ cannot be less than $l$,
since $l$ of the $\omega_\alpha$ roots are zero and to each linear
form $\omega_\alpha$ correspond one or more transformations $X$
such that we have \begin{equation*}
	\bigg[ \sum e_iY_i,X \bigg]=\omega_\alpha X,
\end{equation*}\emph{whatever the coefficients} $e,e_1,\ldots,e_{\lambda-1}$. There
would therefore exist $l-\lambda$ transformations independent of
$Y$ and commuting with $Y$, which is contrary to the hypothesis.\\
\\
The integer $\lambda$ being equal to $l$, and the number of identically
zero $\omega_\alpha$ linear forms not exceeding $l$, it suffices
to give the coefficients $e_i$ numerical values not canceling any
of the non-identically zero $\omega_\alpha$ forms to obtain a \emph{general}
transformation; the transformation given $Y$ is thus part of the
$\gamma$ subgroup defined by this general transformation.\\
\\
\textbf{4.} Any infinitesimal transformation $Y$ is therefore part
of at least one abelian subgroup $\gamma$, containing an infinite
number of general transformations. All the $\gamma$ subgroups are,
moreover, homologous to each other in the continuous adjoint group
$\Gamma$ (\footnote{This is because the \emph{general} infinitesimal transformations $Y$,
each of which defines a $\gamma$ subgroup, form a \emph{connected}
set; indeed, as we will see in a moment, the singular transformations
fill, in the domain of the group, one or more manifolds with 3 dimensions
less than this domain.}), so that any transformation of $\Gamma$ is homologous to a transformation
of a particular $\gamma$ subgroup.\\
\\
That said, let us look at the $l$ fundamental angular parameters
$\phi_1,\phi_2,\ldots,\phi_l$ of an arbitrary infinitesimal transformation
as the Cartesian coordinates of a point in a space of $l$ dimensions.
We will choose the unit vectors of coordinates so that the positive
definite quadratic form \begin{equation*}
	\sum_\alpha \phi_\alpha^2
\end{equation*}represents, to a nearly constant factor, the square, of the distance
from a point to the origin.\\
\\
Any point $M$($\phi_i$) represents an infinitesimal transformation
of each $\gamma$ subgroup, and consequently an infinite number of
homologous infinitesimal transformations between them. If none of
the parameters $\phi_\alpha$ is zero, these transformations are general
and consequently form a set $\infty^{r-l}$. If $h$ of the parameters
$\phi_\alpha$ are zero ($h$ even), each transformation represented
by $M$ is invariant by a subgroup with $l+h$ parameters and therefore
admits $\infty^{r-l-h}$ homologues.\\
\\
Within a given $\gamma$ subgroup, an infinitesimal transformation
admits a certain number of homologues; their representative points
are obtained by carrying out on the $r-l$ parameters $\phi_\alpha$
(regarded as letters) a finite group $\mathcal{G}^\prime$ of substitutions;
these substitutions preserve the linear relations with integer coefficients
which exist between the angular parameters\footnote{There can be substitutions enjoying this property without belonging
to $G^\prime$. See E. Cartan, \emph{Le principe de dualit{\'e} et
la th{\'e}orie des groupes simples et semi-simples}. (Bull. Sc. Math.,
2nd series, t. 49, 1925, pp. 365-366).}. Geometrically the group $\mathcal{G}^\prime$, operating on representative
points $M$, is a group of rotations and symmetries, generated by
$\frac{r-l}{2}$ symmetries with respect to the hyperplanes $\phi_\alpha=0$
(\footnote{This interpretation of the group $\mathcal{G}^\prime$ as a group
of rotations and symmetries, as well as that of its generating operations,
is due to Weyl, \emph{Theorie der Dartellung kontinuierlicher halb-einfacher
Gruppen durch lineare Transformationen}. (Math. Zeitschr., 24, 1925,
pp. 367-371). The group $\mathcal{G}^\prime$ is the group $(S)$
of Weyl.}).\\
\\
If we consider the $\frac{r-l}{2}$ hyperplanes $\phi_\alpha=0$ led
by the origin, they divide the space into a number of undefined regions
(polyhedral angles) convex. Each of them represents the \emph{fundamental
domain} ($D$) of the group $\mathcal{G}^\prime$, and any infinitesimal
transformation of $\gamma$ is homologous to one transformation and
only one inside this region. Any convex region, bounded by a certain
number of $\phi_\alpha=0$ hyperplanes, \emph{and such that no other
of these hyperplanes crosses it}, can be taken as a fundamental domain.
We will verify later that all these regions admit exactly $l$ hyperplane
faces.\\
\\
Any point inside the fundamental domain ($D$) represents $\infty^{r-l}$
homologous infinitesimal transformations; any point located on one
of its faces, or one of its edges, etc., represents at most $\infty^{r-l-2}$
homologous transformations.\\
\\
\textbf{5.} Moving on to finite transformations, or to the $T$ matrices
of the $\Gamma$ group. Let $Y$ be one of its infinitesimal generating
transformations, belonging to a certain $\gamma$ subgroup; we can
represent $T$ and $Y$ by the same point $M$. Now inside the same
$\gamma$ subgroup, the matrix $T$ can be generated by an infinite
number of infinitesimal transformations different from $Y$; they
are those obtained by adding to the fundamental angular parameters
$\phi_i$ arbitrary integers. Let us then consider the lattice ($R$)
of points with integer coordinates $\phi_i$.\\
\\
The same matrix $T$ is represented by an infinite number of points,
homologous among themselves with respect to the lattice ($R$).\\
\\
Suppose that $l+2k$ of the characteristic roots of $T$ are equal
to $1$; we then prove that, $T$ being invariant by a subgroup with
$l+2k$ parameters of $\Gamma$, there exist $\infty^{r-l-2k}$ matrices
homologous to $T$. The hypothesis made amounts to saying that, among
the angular parameters $\phi_\alpha$ of $Y$, $2k$ are integers.
The matrix $T$ can therefore be invariant by a group larger than
its generating transformation $Y$.\\
\\
\textbf{6.} The set of operations of group $\mathcal{G}^\prime$ and
of translations $\mathcal{T}$ which leave the lattice ($R$) invariant
generates a group $\mathcal{G}_1$ of displacements and symmetries.
Two points $M$ homologous with respect to $\mathcal{G}_1$ represent
matrices $T$ homologous with respect to $\Gamma$.\\
\\
We will now consider the set of hyperplanes ($\Pi$) obtained by equating
one of the angular parameters $\phi_\alpha$ to an arbitrary integer.
All these hyperplanes share space in an infinity of convex polyhedra;
let ($P$) be one of them, which we can suppose to be inside the undefined
fundamental domain ($D$). We will show that any matrix $T$ can be
represented by at least one point of ($P$).\\
\\
Let us start from the remark, due to Weyl\footnote{Math. Zeitschr., t. 24, 1925, p. 379.},
that the matrices represented by a given point of the same hyperplane
($\Pi$), admitting at least $l+2$ characteristic roots equal to
$1$, form in the space of the group at most an $r-l-2$ dimensional
manifold; consequently the matrices represented by the different points
of the hyperplanes ($\Pi$) form a finite number of manifolds with
\begin{equation*}
	(r-l-2)+(l-1)=r-3
\end{equation*} dimensions. It is therefore possible to go from one point to another
in the space of the group, i.e. to go continuously from any matrix
$T$ to any other matrix $T^\prime$, avoiding the singular matrices.
Let $M_0$ be a particular point inside ($P$), let $T_0$ be one
of the matrices represented by $M_0$, and let $T$ be any matrix.
By passing from $T_0$ to $T$ in order to avoid singular matrices,
the corresponding representative point, starting from $M_0$, will
remain inside ($P$)\footnote{We can rigorously show that when the matrix $T$ varies in a continuous
way without ever being singular, the representative point $M$ can
also be followed by continuity, without any ambiguity.}, and consequently there exists indeed inside ($P$) a point $M$
representative of $T$. In particular the unit matrix must be represented,
this means that at least one of the vertices of ($P$) belongs to
the lattice ($R$). We can therefore assume, by one of the translations
$\mathcal{T}$, that the polyhedron ($P$) has one of its vertices
at the origin $O$.\\
\\
\textbf{7.} We are now in a position, taking Weyl's general point
of view, to study the topology of the space of the adjoint group $\Gamma$,
in particular to find out whether there exist in this space any firm
contours not reducible to a point by continuous deformation.\\
\\
TODO\\
\\
\textbf{31.} With the isometry group of the simply connected space
$\mathcal{E}$ being transitive, the study of geodesics from any point
is reduced to that of geodesics from the point of origin $O$, which
corresponds to the identical transformation of $G$. These geodesics
correspond to the different one parameter subgroups of $G$; to look
for a geodesic joining $O$ to a given point $A$ is to look for an
infinitesimal transformation generating the finite transformation
represented by $A$. It follows immediately that \emph{by any two
points of $\mathcal{E}$ there always passes a geodesic (and even
an infinity if $l>1$)}. We will leave aside in what follows the case
$l=1$, which corresponds to the three-dimensional spherical space
(and to the elliptical space).\\
\\
Any direction resulting from $O$ represents an infinitesimal transformation
of $G$, which belongs at least to an abelian subgroup $\gamma$ (\S
3). The gamma transformations provide in space $\mathcal{E}$ an $l$-dimensional
manifold $E_l$ passing through $O$. This manifold has zero Riemannian
curvature, since the rotation associated with an elementary parallelogram
whose sides represent the infinitesimal transformations $U$ and $V$
has the effect of giving, at the vector $X$, the geometric increase
$[[U,V],X]$, and is everywhere zero if the bracket $[U,V]$ is zero:
this is the case for any two transformations of $\gamma$.\\
\\
The manifold $E_l$ is therefore locally Euclidean; it is moreover
totally geodesic, since it represents a subgroup of $G$. One can
analytically define a point $A$ of $E_l$ by the $l$ fundamental
angular parameters $\phi_1,\ldots,\phi_l$ of the infinitesimal transformation
$Y$ of $\gamma$ which generates the finite transformation represented
by $A$; the distance $OA$, measured on the corresponding geodesic,
is equal, to a nearly constant factor, to the square root of the sum
$\sum\phi_\alpha^2$ extended to $r-l$ angular parameters. We thus
see that \emph{the representation used in Chapter I in a Euclidean
space with $l$ dimensions is nothing more than an application on
this Euclidean space of the locally Euclidean manifold $E_l$}.\\
\\
There is, however, an important difference. The variety $E_l$ is
not an undefined Euclidean space; if we develop it on the l-dimensional
Euclidean space, it gives the lattice ($\bar{R}$) of parallelepipedes,
\emph{each of which fully represents $E_l$}.\\
\\
We see from this that any point of $E_l$ can be joined to $O$ by
an infinite number of geodesics, located entirely in $E_l$, and which
have for images, in Euclidean space, the lines joining the origin
$O$ to the different homologous points of a point given with respect
to the group $\bar{\mathcal{T}}$ of the translations of the network
($\bar{R}$). Those whose directional parameters are rational are
closed. All the geodesics, located in $E_l$, joining $O$ to $A$
are moreover isolated; none cut itself.\\
\\
\textbf{32.} Instead of representing $E_l$ by the fundamental parallelepiped
of the network ($\bar{R}$), it is better to represent it by the polyhedron
($\mathcal{D}$) introduced in \S 26. The operations of the group
$G^\prime$ correspond to rotations of the continuous isotropy group
of $\mathcal{E}$ which bring the variety $E_l$ into coincidence
with itself, (but without leaving all the points of $E_l$ fixed);
in each of these rotations, the different ($l+1$)-hedra ($P$) in
which the polyhedron ($\mathcal{D}$) is decomposed are transformed
into each other; they are moreover in the same number as the operations
of $G^\prime$ and each of them is a fundamental domain for the discontinuous
group $G^\prime$.\\
\\
Figure 5 thus represents one of the plane manifolds $E_2$ of the
simply connected 8-dimensional space of the simple group of type $A$
of rank 2. The opposite edge of the regular hexagon must be regarded
as identical, their points corresponding to each other by the translation
which brings these two edges into coincidence. The three translations
corresponding to the three pairs of opposite edges generate the group
$\bar{\mathcal{T}}$ of the translations of the lattice ($\bar R$)
(it is basically the holonomy group of the Clifford plane $E_2$ with
respect to the Euclidean plane). The three vertices $O_1$ constitute
only one point of $E_2$; the same is true of the three $O_2$ vertices.
The figure highlights three closed geodesics distended from $O$;
each will pass successively through the points $O_1$ and $O_2$,
which divide it into three equal parts. We have one of these geodesics
starting for example from $O$ in the horizontal direction of the
figure to $O_1$; this geodesic continues with one of the horizontal
dimensions $O_1$ $O_2$, then ends with the horizontal radius $O_2$
$O$ which leads back to the starting point. These three geodesics,
of the same length, intersect in $O$ at angles of $120^\circ$ (if
we take them in the direction which leads first to point $O_1$).
Of course, there is an infinite number of other closed geodesics in
$E_2$, but they are longer.\\
\\
Figure 6 shows the domain ($\mathcal{D}$) representative of a variety
$E_2$ of the simply connected space with 10 dimensions of the simple
unitary group of the type $B$ of rank 2. The opposite sides of the
square correspond by translation and must be considered as identical.
The four vertices of the square represent the same point of $E_2$;
as for the midpoints of the edges, they represent two distinct points
$A$ and $A^\prime$. In the figure there are four closed geodesics
coming from $O$; two of them, which intersect at $O$ at a right
angle, are divided in their middle by the point $O_1$; two others
(whose length is the same as the preceding ones in the ratio $\frac{1}{\sqrt{2}}$)
are divided in their middle, one by point $A$, the other by point
$A^\prime$. There are also in the figure two other closed geodesics
passing through $O_1$ and shared in their middle, one by $A$, the
other by $A^\prime$.\\
\\
FIG. 7 represents a variety $E_2$ of the space with 14 dimensions
of the simple group of type $G$. The vertices of the regular hexagon
represent two distinct points $A$ and $A^\prime$ of $E_2$; the
midpoints of the edges represent three other distinct points $B$,
$B^\prime$, $B^{\prime \prime}$. The figure represents three closed
geodesics originating from $O$ and passing successively through points
$A$ and $A^\prime$ which divide them into three equal parts; they
are on the other hand cut in their middle by one of the points $B$,
$B^\prime$ or $B^{\prime \prime}$. The figure shows three other
closed geodesics cut in their middle by one of these three points,
but passing neither through $A$ nor through $A^\prime$. The compared
lengths of these two types of geodesics are $3$ and $\sqrt{3}$.\\
\\
\textbf{33.} Let us return to the general study of geodesics. If a
given point $A$ other than $O$ belongs to a single manifold $E_l$,
all the geodesics joining $O$ to $A$ are in this manifold; they
are isolated and there is an infinite number of them.\\
\\
If the point $A$ belongs to an infinity of manifolds $E_l$, the
finite transformation of $G$ represented by $A$ is exchangeable
with more than $l$ independent infinitesimal transformations, that
is to say $l+\lambda$. We will have the number $\lambda$ by looking
for how many, among the $r-l$ angular parameters $\phi_\alpha$ of
$A$, are integers. The point $A$ is then invariant by a subgroup
$g_{l+\lambda}$ of the continuous group of rotations around $O$.
It therefore belongs to $\infty^\lambda$ distinct $E_l$ manifolds,
and in each of them there exists a countably infinite number of geodesics
joining $O$ to $A$. Suppose that the direction in $O$ of one of
these geodesics is invariant by a subgroup $g_{l+\lambda}$ of the
group of rotations around $O$; the corresponding geodesic will belong
to $\infty^\mu$ distinct $E_l$ varieties. If $\mu=\lambda$, this
geodesic will be isolated in space $\mathcal{E}$ (among the set of
geodesics joining $O$ to $A$. If $\mu<\lambda$, this geodesic will
belong to a continuous variety with $\lambda-\mu+1$ dimensions of
geodesics joining $O$ to $A$, and this variety will be obtained
by applying to the given geodesic all the rotations of the group $g_{l+\lambda}$
which leaves fixed the points $O$ and $A$.\\
\\
Finally, \emph{certain geodesics joining $O$ to $A$ may not be isolated};
this case arises if the group of rotations which leave the point $A$
invariant is greater than the group of rotations which leave the direction
in $O$ of the geodesic invariant; the dimension of the continuous
manifold of which geodesics are a part is the difference increased
by 1 between the orders of these two groups.
\end{document}
